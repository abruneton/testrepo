\chapter{Introduction and overview}
\label{ch:intro}

\section{Freeform optics, a key player in energy efficiency}
In the past two decades the LEDs (\emph{Light Emitting Diodes}) have
imposed themselves as the next generation of light sources, covering a 
wide range of application from coherent light emission in space
experiments (\TODO bad example - coherent!) to daily use as flash lights in cell-phone embedded cameras. 
Their development was driven by several drawbacks of the old-generation 
light sources (mainly the incandescent bulbs), on top of which one finds the wall-plug efficiency (how much
light power you obtain for one unit of electricity power). A typical light 
incandescent light bulb produces hardly 5\% of visible light, the rest being wasted
as infra-red light, or direct heat. This energy saving constraint is now enforced 
legally in many countries around the globe with a ban on incandescent light bulbs.

In lighting applications the same energy-saving motivation has led researchers and 
key actors of the industry to also inspect and question the pure optical efficiency 
of their products. As it appeared, in many lighting devices the light itself is 
not used sparingly, a significant part of the flux being lost between the source
and the final target. Standard optical elements (mostly spherical lenses and conic
reflectors) are
difficult to combine in such a way that the full light cone emitted by the 
source is optimally used and brought onto the target area to be illuminated. For 
example in old street light devices, where 
the reflector part has a simple design, less than 50\% of the light flux produced
by the source is reaching the intended area on the street.

The \emph{freeform optics} were invented in response to this challenge 
and further pushed forwards by an increasing need of
precise light-tailoring applications. The youth of this field of research
makes a precise definition of what is a freeform optics difficult to give, 
except that they are not conventional optics! A suitable approach(\TODO cite Axel?) however is to 
consider the degrees of freedom needed to describe such an element. Standard optics
have typically few degree of freedoms: spherical lenses are rotation symmetric, and 
three geometrical parameters (diameter, thickness and focal length) suffice 
to describe them fully. On the other side of the spectrum, an optical 
surface requiring an exhaustive sampling of several 

\TODO mention solar concentrators

Freeform optics thus share the characteristic of requiring a complex rich
parametrisation and of being (mostly) non-imaging devices (in opposition with 
a camera objective for example).  

The design of freeform optics is thus becoming the preferred route to
achieve compact optical systems producing a prescribed irradiance or
radiant intensity distribution.  They find a wide scope of applications ranging
from general lighting (street lighting for example) to automotive
lighting and even more specialized applications like laser beam
shaping.

%A precise analysis of those limitations will be the subject of another
%article by the authors.
%Based on those
%findings, and still for a null-\'{e}tendue source, we therefore
%propose an extended algorithm that enables improved homogeneity and
%higher achievable resolutions within the target irradiance patterns.
%
%In the first part of this article, the principle of the ray mapping
%computation as described in~\cite{Baeuerle2012} is briefly reviewed.
%The main contribution of this article is then presented in the
%subsequent section: the parametrization chosen for the surface
%representation allows to readily compute a localized light flux
%traversing the optical elements. A comparison of the flux hitting the
%target with the prescribed irradiance allows the optics to be directly
%optimized to fine-tune the results obtained from the mapping.
%Important numerical aspects have to be taken into account during
%calculations and will be highlighted.

\section{Market trend - LED is the new light source}



\section{Typical applications requiring freeform optics}
Guiding the light from a radiant source to a given target 

\img AUTOLIGHT?

Not to be forgotten are all the applications where the inverse situation is 
found: the source is not under the control of the designer, who then
tries to collect the light from an external system.
This is typically the case of the solar concentrators where the light of the sun
is to be brought on a small area containing a photovoltaic cell or a thermal device
to be heated.
This application was by the way a major driver in the development of
freeform reflectors (\cit Non imaging 
optics book).

\section{Formal statement of the problem}

Historically two set of terms can be used to describe the various physical quantities. 
The photometric community speaks of luminance, illuminance, and luminous intensity, while
the radiometric community speaks of radiance, irradiance and radiant intensity. The key 
difference between the two is that the former uses the lumen ($lm$) as the primary
 light power unit, while the second simply uses the watt ($W$).
The above is mostly not relevant for the work presented here, and we hence try to stick 
to the photometric terminology. The reader is directed toward the glossary for a more
precise definition of the terms. 

Although freeform optics start to be used in conjunction with coherent light 
sources, most of the applications cited above only requires geometrical 
optics. We hence assume a Fresnel number much greater than unity, and use only
the laws of geometrical optics, i.e. Snell's laws of reflection and refraction.

\img from the seminar, or from the article with the flux tubes

\section{Contribution of the author to the field}

link with the Monge problem and use of Haker's evolution equation to solve it
integrability conditions detailing the area of validity of the above heuristic
flux optimization
extended source consideration in 2D 

\section{Structure of this work}
\TODO at the very end.


