\chapter{Introduction}
\label{ch:intro}


\section{What freeform optics boil down to}

More and more lighting applications require the design of dedicated
optics to achieve a given radiant intensity or irradiance distribution.
Freeform optics has the advantage of providing such a functionality
with a compact design.  It was previously demonstrated in
[B{\"{a}}uerle et al., Opt. Exp.  20, 14477-14485 (2012)] that the
up-front computation of the light path through the optical system (ray
mapping) provides a satisfactory approximation to the problem, and
allows the design of multiple freeform surfaces in transmission or in
reflection.  This article presents one natural extension of this work
by introducing an efficient optimization procedure based on the
physics of the system.  The procedure allows the design of multiple
freeform surfaces and can render high resolution irradiance patterns,
as demonstrated by several examples, in particular by a lens made of
two freeform surfaces projecting a high resolution logo ($530\times
160$ pixels).


The design of freeform optics is becoming the preferred route to
achieve compact optical systems producing a prescribed irradiance or
radiant intensity distribution.  They find a wide scope of applications ranging
from general lighting (street lighting for example) to automotive
lighting and even more specialized applications like laser beam
shaping.

In a previous publication~\cite{Baeuerle2012}, the authors present a
freeform design algorithm that can handle multiple optical surfaces
whilst at the same time operating directly with a prescribed target
irradiance pattern. In its practical implementation, this approach has
some limitations with respect to the achievable resolution and the
precision of the generated lighting patterns, in particularly for more
elaborate and highly asymmetrical geometries.  
A precise analysis of those limitations will be the subject of another
article by the authors.
Based on those
findings, and still for a null-\'{e}tendue source, we therefore
propose an extended algorithm that enables improved homogeneity and
higher achievable resolutions within the target irradiance patterns.

In the first part of this article, the principle of the ray mapping
computation as described in~\cite{Baeuerle2012} is briefly reviewed.
The main contribution of this article is then presented in the
subsequent section: the parametrization chosen for the surface
representation allows to readily compute a localized light flux
traversing the optical elements. A comparison of the flux hitting the
target with the prescribed irradiance allows the optics to be directly
optimized to fine-tune the results obtained from the mapping.
Important numerical aspects have to be taken into account during
calculations and will be highlighted.

\section{Market trend - LED is the new light source}

\section{Typical applications requiring freeform optics}

\section{Underlying physics ( appendix?)}

Definition of physical units
Geometrical optics
Fresnel losses
