\chapter{Introduction and overview}
\label{ch:intro}

\section{Freeform optics, a key player in energy efficiency}
In the past two decades the LEDs (\emph{Light Emitting Diodes}) have
imposed themselves as the next generation of light sources, covering a 
wide range of application from coherent light emission in space
experiments (\TODO bad example - coherent!) to daily use as flash lights in cell-phone embedded cameras. 
Their development was driven by several drawbacks of the old-generation 
light sources (mainly the incandescent bulbs), on top of which one finds the wall-plug efficiency (how much
light power you obtain for one unit of electricity power). A typical light 
incandescent light bulb produces hardly 5\% of visible light, the rest being wasted
as infra-red light, or direct heat. This energy saving constraint is now enforced 
legally in many countries around the globe with a ban on incandescent light bulbs.

In lighting applications the same energy-saving motivation has led researchers and 
key actors of the industry to also inspect and question the pure optical efficiency 
of their products. As it appeared, in many lighting devices the light itself is 
not used sparingly, a significant part of the flux being lost between the source
and the final target. Standard optical elements (mostly spherical lenses and conic
reflectors) are
difficult to combine in such a way that the full light cone emitted by the 
source is optimally used and brought onto the target area to be illuminated. For 
example in old street light devices, where 
the reflector part has a simple design, less than 50\% of the light flux produced
by the source is reaching the intended area on the street.

\section{Typical applications requiring freeform optics}
The \emph{freeform optics} were invented partly in response to this challenge 
and further pushed forwards by an increasing need of
precise light-tailoring applications (see an example below of an automotive
application \ref{intro:img_auto}). The youth of this field of research
makes a precise definition of what is a freeform optics difficult to give, 
except that they are not conventional optics. A suitable approach(\TODO cite Axel?) however is to 
consider the degrees of freedom needed to describe such an element. Standard optics
have typically few degree of freedoms: spherical lenses are rotation symmetric, and 
three geometrical parameters (diameter, thickness and focal length) suffice 
to describe them fully. On the other side of the spectrum, an optical 
surface requiring an exhaustive sampling of several points to be precisely described
can certainly be seen as a freeform optics.


\img AUTOLIGHT?


Another key driver of the developments in the domain can be understood if one sees
the problem the other way around. Instead of producing 
a predefined pattern from a light source,
solar concentrators aim at collecting as efficiently as possible the light from a 
remote ''wide'' source (what is understood by ''wide'' will be later precised in
this work with the concept of �tendue) to bring it onto a small device (typically a heat 
tube or a solar cell). There again ''simple'' optics have shown their limitations and the
need appeared to produce a specific optic design for such an application.

OR OR : Not to be forgotten are all the applications where the inverse situation is 
found: the source is not under the control of the designer, who then
tries to collect the light from an external system.
This is typically the case of the solar concentrators where the light of the sun
is to be brought on a small area containing a photovoltaic cell or a thermal device
to be heated.
This application was by the way a major driver in the development of
freeform reflectors (\cit Non imaging 
optics book). The present work won't present any application in this field, but the 
domain remains however very proeminent.

Freeform optics thus share the characteristic of requiring a complex, rich
parametrisation and of being (mostly) non-imaging devices (in opposition with 
a camera objective for example).  

The design of freeform optics is becoming the preferred route to
achieve compact optical systems producing a prescribed irradiance or
radiant intensity distribution.  They find a wide scope of applications ranging
from general lighting (street lighting for example) to automotive
lighting and even more specialized applications like laser beam
shaping.

%A precise analysis of those limitations will be the subject of another
%article by the authors.
%Based on those
%findings, and still for a null-\'{e}tendue source, we therefore
%propose an extended algorithm that enables improved homogeneity and
%higher achievable resolutions within the target irradiance patterns.
%
%In the first part of this article, the principle of the ray mapping
%computation as described in~\cite{Baeuerle2012} is briefly reviewed.
%The main contribution of this article is then presented in the
%subsequent section: the parametrization chosen for the surface
%representation allows to readily compute a localized light flux
%traversing the optical elements. A comparison of the flux hitting the
%target with the prescribed irradiance allows the optics to be directly
%optimized to fine-tune the results obtained from the mapping.
%Important numerical aspects have to be taken into account during
%calculations and will be highlighted.

\section{Physical units - Etendue}
The physical \emph{\etendue} characterizes how ''spread'' a source is, how ''not-punctual'' it
is. Said differently this determines how well one can theoritically focus the light from such a source 
to a point. 
Let's take the example of a light source in form of a disc, whose every point on the surface emits light
within a given angular range around the normal. Intuitively we see that no matter how hard one tries, the
light of such a source can not be all brought to a single point on a screen. The \etendue expresses
this physical limitation.

Formally the \etendue is defined as a double integral over the 2D surface of the emitter:

\[
	d\Omega = \iint{dS}
\] 

A source with a null \etendue has the advantageous feature that a point in space is only 
met by one single light ray emitted by the source. An extended source (with a non-zero \etendue)
violates this simple property. 

We recall finally that the concept of \etendue is closely related to the one of phase-space, 
which we will not cover
extensively in this work. The \etendue is direclty proportional to the volume of the 
representation of the source in the phase-space. Manipulating phase-space representations provides 
a great analysis tool when having already defined an optical system, but does not help much 
when dealing with the design process itself.

\section{Formal statement of the problem}

Historically two sets of terms can be used to describe the various physical quantities. 
The photometric community speaks of luminance, illuminance, and luminous intensity, while
the radiometric community speaks of radiance, irradiance and radiant intensity. The key 
difference between the two is that the former uses the lumen ($lm$) as the primary
light power unit, while the second simply uses the watt ($W$). The lumen is directly related to the 
way the human eye perceive the light, while the watt is a standard physical unit of power.
The above is mostly not relevant for the work presented here, and we hence try to stick 
to the photometric terminology. The reader is directed toward the glossary for a more
precise definition of the terms. 

Although freeform optics start to be used in conjunction with coherent light 
sources, most of the applications cited above only requires geometrical 
optics. We hence assume a Fresnel number much greater than unity, and use only
the laws of \emph{geometrical} optics, i.e. Snell's laws of reflection and refraction.
As we will see, this sets up already leads to a complex formulation of the problem.

\img from the seminar, or from the article with the flux tubes

Given a light source $S$, a target surface $T$, and a target irradiance $I$, one seeks the optical surface 
$\Sigma$ turning the luminous intensity of the source into the desired irradiance on the 
target. To be clear, no masking element is to be used, that is to say one wishes to achieve $I$ just by
deflecting the light . Said differently, if we assume no power loss (due to refraction or absorption of
the optical element), the total light power emitted by the source must be found on the target.
The problem can be considered in a two-dimensional space, or in three dimensions, the latter posing much 
more difficulties, as will be further explained.

\section{Contribution of the author to the field}

The first section of the present work highlights the previous contributions to the field.
The author's contribution to this area of research can be stated as follow:
\begin{list}
\item in conjunction with A. Ba�erle (\cite{DissAxel}), the author recalls the link between this problem
and the Monge's optimal mass transport problem. The use of Haker's evolution equation to provide
an efficient heuristic to solve the problem was new to the field, and has proven to be a valid
solution in many practical usages.
\item the author formally demonstrates the adequation between the heuristic mentioned above and the exact
solution of the problem, and proposes quantitative criteria to assess the quality of the method.
\item a direct optimisation procedure on the flux received by the target is then presented, and combined 
with the above allows to refine the result up to a high level of quality, as demonstrated on a high-resolution
example
\item the author then suggests an extension of the approach to extended sources. 
For the two dimensional problem, an integration scheme is proposed and results are presented validating 
the approach. The basis of a similar reasoning for the 3D case are given.
\item ?? segmented optics ??
\end{list}

\section{Structure of this work}
The present work is structured as follow. In the first chapter the state of the art is reviewed. In a second 
chapter the ray mapping procedure published previously is detailled further, as it serves as basis
for the rest of the work.
The integrability conditions and their range of validity are then analyzed in a third chapter, with a
quantitative criteria precised. 
Finally the problem of real extended source is being adressed, and a numerical solution is being described
for the two dimensional case. 

Many example applications will be given along the chapters, illustrating the point being made.

\TODO at the very end.


