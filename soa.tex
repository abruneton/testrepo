\chapter{State of the art}
\label{ch:soa}

\section{The freeform algorithms at hand - An overview}
Several procedure have been described in the litterature to compute 
freeform optics. We review here the most relevant ones, and highlight
their respective advantages and drawbacks.

We also covers the literature dealing with extended source since this 
forms a major part of this work.


Previous work on high-resolution irradiance tailoring was most notably
performed by Ries and Muschaweck~\cite{Ries2002}, who outlined a
single-step method that minimizes the deviation between the prescribed
and the realized irradiance pattern on the target. For a given surface
shape, the irradiance realized on the target is derived from the
curvature tensor of the outgoing wave field, which in turn is computed
using the curvature tensors of the incoming wave-front and that of the
optical surface. The computation of the curvature tensor of the
optical surface involves the second derivatives of its parametrization
and the resulting equations are highly non-linear. Convergence to the
global minimum requires the initial solution to be sufficiently close
to the global minimum. One way to achieve this is using a multi-grid
technique and a good first guess~\cite{Baeuerle2010}.

The main advantage of this method is that the optical surface is
readily represented by a smooth function, as the field of surface
normals automatically conforms to the integrability
condition~Eq.~\eqref{eq:integ}. However, a disadvantage is the high
computational cost which is partly due to the sensitivity of the
target irradiance to the surface's second derivatives. In addition,
this procedure has, to the best of the authors' knowledge, yet only
been published and proven for a single optical surface, which is a
major difference with the work presented in~\cite{Baeuerle2012} and
here.



\section{Analytical methods}
\subsection{Schruben and other functional methods}

\subsection{Curvature tensor method - Ries}

\subsection{Conics methods - Oliker}

\section{Algorithmical methods}
\subsection{Brute force ray tracing (??) }
\subsection{Simultaneous Multiple Surfaces - Benitez}


\section{Ray mapping computation and optimization - Axel's part}
\subsection{Projection and initial mapping computation}

\subsection{Haker's evolution equation}




