\chapter{State of the art}
\label{ch:soa}

\section{The freeform algorithms at hand - An overview}
Several procedure have been described in the litterature to compute 
freeform optics. The most relevant ones are reviewed here, and we highlight
their respective advantages and drawbacks.

We also covers the literature dealing with extended source since this 
forms a major part of this work.

The separation is made between analytical methods, which tackle the issue in a formal
direct mathematical way (which does not always lead to a practical construction
method of the optical surface), and the algorithmical methods, which are more
constructive by nature, although not always exact.

At the end of this chapter, an entire section is dedicated to the work done in conjunction
with A. Ba�erle
on the ray mapping approach. This is indeed paramount to the understanding of the rest of this 
work.

\section{Analytical methods}
\subsection{Schruben and other functional methods}

\subsection{Curvature tensor method - Ries}
Ries and Muschaweck~\cite{Ries2002} outlined a
single-step method that minimizes the deviation between the prescribed
and the realized irradiance pattern on the target. For a given surface
shape, the irradiance realized on the target is derived from the
curvature tensor of the outgoing wave field, which in turn is computed
using the curvature tensors of the incoming wave-front and that of the
optical surface. The computation of the curvature tensor of the
optical surface involves the second derivatives of its parametrization
and the resulting equations are highly non-linear. Convergence to the
global minimum requires the initial solution to be sufficiently close
to the global minimum. One way to achieve this is using a multi-grid
technique and a good first guess~\cite{Baeuerle2010}.

The main advantage of this method is that the optical surface is
readily represented by a smooth function, as the field of surface
normals automatically conforms to the integrability
condition~Eq.~\eqref{eq:integ}. However, a disadvantage is the high
computational cost which is partly due to the sensitivity of the
target irradiance to the surface's second derivatives. In addition,
this procedure has, to the best of the authors' knowledge, yet only
been published and proven for a single optical surface, which is a
major difference with the work presented in~\cite{Baeuerle2012} and
here.


\subsection{Conics methods - Oliker}
In standard optics, the geometrical properties of conics (parabola, ellipse) have used since
the antiquity. A perfect point light source  placed at the (\bf{FOYER!!}) of a parabola produces
a perfectly collimated flux of light. Similarly all rays emitted from the FOYER of an ellipse are 
focused onto the second FOYER. 
Oliker suggests to build the freeform surface as a collection of section of conics directing
part of the light flux emitted by the point source to a given point on the target. The orientation
of the conical section dictated the direction taken by light and hence the area hit on the target, 
and the area of the section dictates the relative irradiance of the area being illuminated.
Oliker shows formally that with an infinite collection the section thus obtained is smooth and fulfills 
the problem. 
Practical applications of this technique are however limited by the complexity of the procedure, which
at the time of writing is quadratic with the number of conic patches being used, and by the fact that
with a limited number of such patches, the constructed surface is no continuous, but not $C_1$ (edges
are seen at the conection lines between patches).

\section{Algorithmical methods}
\subsection{Brute force ray tracing (??) }
\subsection{Simultaneous Multiple Surfaces - Benitez}

Benitez proposes a constructive method to solve the problem. 
The entry point of the process consists in two \emph{ray congruences} (an
orthonormal ray bundle, i.e. a ray bundle emitted by a source with 
null \etendue) to be coupled through an optical surface.

<re read and describe>

When designing optical systems to achieve a given target irradiance, the key challenge in the
process is to convert the irradiance description into a ray congruence formulation.
Benitez' method has proven however very successful in many applications (cite cite cite!)

\section{Ray mapping computation and optimization - Axel's part}
\subsection{Projection and initial mapping computation}
< a pomper depuis les articles>
\subsection{Haker's evolution equation}
< a pomper depuis les articles>



